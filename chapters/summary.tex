\documentclass[../main]{subfiles}
\begin{document}

\newpage
\chapter*{結言}
\addcontentsline{toc}{chapter}{結言}
\label{chap:summary}
ヒトの感覚では, 聴覚より視覚の方が情報量が多く, 
視覚を失うと普段の生活に大変な支障が伴うことが想像できる.
コウモリはヒトと同じ哺乳類でありながら, 
ヒトには真似できないほど巧みに
聴覚情報を扱い, 行動を制御している.
本研究では, コウモリが聴覚情報をどうやって処理しているかに
関しては扱わず, 
視覚と比較すると情報が限定的と考えられる
エコーロケーションによって,
コウモリに似た行動が可能なのかという視点で研究を行った.

聴覚情報から計算可能と考えうる空間の特徴量を, 
コウモリ型のエージェントに与え, 
生物の学習能力から着想を得た学習アルゴリズムである 
強化学習によって, 回避行動をシミュレーション上で学習させた.
エージェントの振る舞いには, 
コウモリと類似している点と, そうでない点が存在するが, 
コウモリが空間学習によって, 
消費エネルギーを最小化しているという仮説を
支持することができたと考える.

今後も, 行動実験, 強化学習両面からコウモリの行動解析をすることで, 
コウモリの行動決定因子を解明することが期待できる.
また, 本研究での強化学習を用いた行動解析の手法は, 
コウモリに限らず, 様々な動物の行動理由を解き明かす
可能性を持っていると考える.

\end{document}