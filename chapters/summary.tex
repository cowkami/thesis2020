\documentclass[../main]{subfiles}
\begin{document}

\newpage
\chapter*{結言}
\addcontentsline{toc}{chapter}{結言}
\label{chap:summary}
多くのヒトは普段の生活で, 
視覚から得られる情報に強く依存して生活している.
コウモリはヒトと同じ哺乳類でありながら, 
ヒトには真似できないほど巧みに
聴覚情報を扱い, 行動を制御していて, 
ヒトの感覚では信じられないほど複雑な行動をみせる.
では, なぜコウモリはそのような行動をするのか.
本研究では, コウモリの行動理由を解明することを目的に, 
行動実験から得られたデータから, 
コウモリの行動理由を説明する方法を考えた.
そこで, コウモリがどんな行動に価値を感じているかを推測し, 
聴覚情報から計算可能と考えうる空間の特徴量を, 
コウモリ型のエージェントに与え, 
回避行動をシミュレーション上で学習させた.
様々なパラメータを検討した結果, 
エージェントはコウモリの回避飛行に類似した行動を
学習することに成功した.また, 
このパラメータから, 
コウモリの行動則として仮定した, 
"空間学習による消費エネルギー最小化"
という行動則が実物のコウモリに備わっていることが示唆された.

また, 行動実験との相違点からは, 
コウモリはより高度で複雑なセンシングをしていることが示唆された.

今後も, 行動実験, 強化学習両面からコウモリの行動解析をすることで, 
空間学習だけでなく, その他のコウモリ特有のユニークな行動が
どのように起きるのか, 
また, なぜ起きるのかに関して解き明かすことが期待できる.
また, 本研究での強化学習を用いた行動解析の手法は, 
コウモリに限らず, 様々な動物の行動理由を解き明かす
可能性を持っていると考える.

\end{document}