\documentclass[../main]{subfiles}
\begin{document}

\newpage
\chapter{シミュレーション実験}
\label{chap:simulation}

\section{環境とエージェント}
シミュレーションと比較対象となる行動実験では,
Fig.\ref{fig:top_view_measure}
に示す行動空間($4.5 \times 1.5$\si{m}四方)を用いた.
コウモリを図の位置から,複数個体,複数回飛行させ,
コウモリの飛行軌道とパルス放射方向を記録したデータを用いた.

シミュレーションでは, 行動実験の空間と同じ比率になるように2次元空間を設定した.
(Fig.\ref{fig:simulation_field})
エージェントはこの空間内を自由に移動できるが,
空間内の実線が壁に対応しており,
実線を交差するような移動はできない.
そのような移動が選択された場合, 
-100の報酬がエージェントに与えられ,その試行は終了する.

エージェントの初期位置は図中に示した位置を中心に,
上下左右に0.05\si{m}の範囲で,各試行毎にランダムに出現する.
これは,行動実験で生じる,コウモリの飛行開始位置のばらつきを再現するためである.

エージェントは行動開始から, 毎ステップ進行方向に決められた距離だけ前進する.
その際,同時に以下の2つの行動を行うことができる.
\begin{itemize}
    \item 進行方向の変更
    \item パルス放射
\end{itemize}
2つの行動に関わるパラメータは,
方向変更角$\Delta\theta_t$,
パルス放射確率$p^{emit}_t$,
パルス放射角度$\phi_t$の3つで,これらをひとまとめに,
$$a_t=\{\Delta\theta_t, p^{emit}_t, \phi_t \}$$
と表記することにする.
各パラメータのとる値の範囲をTable.\ref{tab:agent actions}にまとめた.
ここで,$\Delta\theta_t, \phi_t$はエージェントの進行方向を0とした,
相対角度である.


\subsection{進行方向の変更}
時刻$t$におけるエージェントの進行方向を$\theta_t$とする.
このとき進行方向の変更は次式によって更新される.
$$\theta_{t+1}=\theta_t+\Delta\theta_t$$

\subsection{パルス放射}
パルス放射



\section{実験結果}
\section{考察}
なんもわからん!

\end{document}