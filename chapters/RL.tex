\documentclass[../main]{subfiles}
\begin{document}

\newpage
\chapter{強化学習}
\label{chap:RL}

強化学習では, ある制御対象のシステムを環境,
その環境内での制御者のことをエージェントと呼ぶ.
強化学習の目的は, エージェントが環境内で最適な方策を求めることである.
エージェントは環境に対する情報をほとんど持っておらず,
環境内で行動を繰り返していくことで,
自ら環境のデータを収集し, 最適な方策を探索していく.

\section{マルコフ決定過程}
エージェントはある時刻$t$において,環境から状態$s_t$を得る.
環境に対して方策$\pi(\cdot|s_t)$に従って,
行動$a_t$を起こすことで報酬関数$g(s_t,a_t)$による報酬$r_t$と, 
次の状態$s_{t+1}$が得られる. 
報酬を最大化する最適な方策を求めることが強化学習の目的である.


本実験では,強化学習アルゴリズムとして,
PPO(Proximal Policy Optimization)\cite{ref:proximal_policy}を用いた.

\end{document}