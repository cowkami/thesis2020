\documentclass[../main]{subfiles}
\begin{document}

\newpage
\chapter{考察}
\label{chap:result}

\section{考察}
本研究では, 報酬設計を含む, 適切なシミュレーション環境において, 
強化学習によって獲得される方策に基づくエージェントの行動は, 
実物のコウモリの行動と類似した結果になることが期待されている.
そのため, エージェントの行動と, 行動実験の結果を比較し, 
類似点と相違点に注目して考察を行った. 
また, シミュレーションをより現実に近づけるために
必要となる要素や工夫について検討した.

\subsection{軌道}

\subsection{パルス放射回数}

\subsection{パルス放射方向}

\section{まとめ}
エージェントの行動に関して, 飛行軌跡, 
パルス放射回数の変化は行動実験と同様の傾向が現れた.
この結果により,
コウモリも空間の学習が進むにしたがって
より消費エネルギーを減少させる行動を選択していると考えられる.
対して, パルス放射角度は行動実験とは異なり, 
シミュレーションでは飛行方向から大きな角度をつけ放射する傾向となった.
これより, コウモリは, シミュレーションの設定のように, 
最近傍点1点のみを見ているのではなく.
放射パルスの指向性をうまく活用し, 
環境把握ができているのではないかと推測される.

    
\end{document}