\documentclass[../main]{subfiles}
\begin{document}


\newpage
\chapter*{緒言}
\addcontentsline{toc}{chapter}{緒言}

コウモリは, 超音波を放射し, 周囲からのエコーを脳内で処理することで
空間情報を高度に取得している(エコーロケーション行動).
このコウモリが日常的に行っているエコーロケーション行動は, 
音響計測を通じた行動実験や, 
ボトムアップ的に行われている神経生理学的実験等により, 
生物学的のみならず, 
工学的にもユニークであり優れていることが解明されてきている
\cite{ref:bat_enhance}\cite{ref:echolocating_bats}.
しかし従来からのこれらの実験では, 
コウモリが与えられた環境に対して, 
どのような行動を示すかを計測することはできるが, 
“なぜそのような行動を選択するのか”という, 
コウモリの意思判断の基準やルールは推測するしかなかった.

一方で, 近年発展が著しい機械学習分野の生物学分野への融合が図られ始め, 
特に報酬と行動との関連を学習する強化学習・逆強化学習を
生物の行動研究に応用する例も出始めている
\cite{ref:simulating_bout}\cite{ref:can_ai}.
これらの手法を用いた動物の行動解析により, 
従来は推測に留まっていた仮説を支持する結果が期待される.

本研究では, コウモリの飛行行動を決定する行動因子の
解明を目的に, コウモリの飛行実験を模擬したシミュレーション実験を行った.
飛行実験では, コウモリが繰り返し同じ空間を飛行することで, 
空間を学習し, 飛行行動を最適化していることが示唆された.
この実験環境とコウモリを模擬したシミュレーションを作成し, 
強化学習によって最適な飛行方策を学習する.
設定した報酬によって, エージェントのとる行動を検討することで, 
コウモリが重要視しているパラメータを推定し, 
実際の飛行実験により得られた行動と比較した.
各パラメータの重要度を解明することにより, 
コウモリの効率的な飛行を理解するだけではなく, 
エコーロケーションによる, 
ドローンの自律飛行アルゴリズムへの転用, 
などの工学的な応用が期待できる.

本論文は3章構成であり, 
第\ref{chap:RL}章では, 
強化学習に関する理論と, 
実際に用いた強化学習アルゴリズムについて記述した.
第\ref{chap:simulation}章では, 
本研究の概要, 実験方法, 結果について記述した.
第\ref{chap:result}章では, 
考察とまとめについて記述した.
最後に結言では, 本研究の総括と, 
今後の展望, 課題について記述した.


\end{document}