\documentclass[../main]{subfiles}
\begin{document}

\newpage
\chapter{序論}

\section{研究背景}
コウモリは,超音波を放射し,周囲からのエコーを脳内で処理することで
空間情報を高度に取得している(エコーロケーション行動).
このコウモリが日常的に行っているエコーロケーション行動は,
音響計測を通じた行動実験や,
ボトムアップ的に行われている神経生理学的実験等により,
生物学的のみならず,
工学的にもユニークであり優れていることが解明されてきている
\cite{ref:bat_enhance}\cite{ref:echolocating_bats}.
しかし従来からのこれらの実験では,
コウモリが与えられた環境に対して,
どのような行動を示すかを計測することはできるが,
“なぜそのような行動を選択するのか”という,
コウモリの意思判断の基準やルールは推測するしかなかった.

一方で,近年発展が著しい機械学習分野の生物学分野への融合が図られ始め,
特に報酬と行動との関連を学習する強化学習・逆強化学習を
生物の行動研究に応用する例も出始めている
\cite{ref:simulating_bout}\cite{ref:can_ai}.


\section{研究目的}
本研究では,コウモリの飛行とパルス放射方向を決定する環境因子の
同定を目的に,コウモリの飛行実験を模擬したシミュレーション実験を行った.
コウモリを模擬したエージェントは強化学習によって最適な飛行方策を学習する.
設定した報酬により,どのような行動を選択するのかを検討することで,
コウモリが重要視しているパラメータを推定し,
実際の飛行実験により得られた行動と比較した.


\end{document}